\documentclass{article}

\begin{document}

\title{Introduction to probability distribution}

\maketitle


\section{Starters}
\begin{itemize}
	\item When  the random variable is discrete in nature, its probability distribution is characterized by \textbf{Probability Mass Function (PMF)}.
	\begin{center}
	\begin{tabular}{ |c|c|c| } 
 	\hline
 	\textbf{No. of fruits sold} & \textbf{no. of customers} & \textbf{PMF} \\ 
 	\hline
 	3 & 30 & 30/60 \\ 
 	5 & 20 & 20/60 \\ 
 	7 & 10 & 10/60 \\ 
 	\hline
 	  & 60 & sum = 1 \\ 
 	\hline
	\end{tabular}
	\end{center}
	\item A \textbf{Cumulative Distribution Function (CDF)} defines the less than, greater than or equal to argument of a function.
	
	CDF is a monotonic increasing function.
	
	For the above PMF, CDF of $P(X \le x)$ could be calculated as -
	\begin{center}
	\begin{tabular}{ |c|c|c|c| } 
 	\hline
 	 	\textbf{No. of fruits sold} & \textbf{no. of customers} & \textbf{PMF} & \textbf{CDF}\\ 
 	\hline
 	3 & 30 & 30/60 & 30/60\\ 
 	5 & 20 & 20/60 & (30/60) + (20/60)\\ 
 	7 & 10 & 10/60 & (30/60) + (20/60) + (10/60)\\ 
 	\hline
 	  & 60 & sum = 1 & last value itself becomes 1\\ 
 	\hline
	\end{tabular}
	\end{center}
	\item  When  the random variable is continuous in nature, its probability distribution is characterized by \textbf{Probability Density Function (PDF)}.

\end{itemize}

\section{Discrete PD}
\textbf{Topics}: Uniform | Binomial | Negative binomial | Poisson

\subsection{Uniform}
\begin{itemize}
	\item A random variable which assumes equal probability for its outcomes, is termed as discrete uniform PD.
	
	e.g. getting 5 in a throw of a dice. Same goes with other number on dice.
\end{itemize}

\subsection{Binomial}
\begin{itemize}
	\item 
\end{itemize}




\end{document}